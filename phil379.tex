% Open Logic Project
%
% driver file open-logic-sample.tex to produce text on letter-size paper
% with standard layout and margins

% We use the memoir class for maximal flexibility of layout, but any
% class will do

\documentclass[letterpaper]{memoir}

% \olpath has to point to the location of the OLP main
% directory/folder.  We're compiling from subdirectory courses/sample,
% so the main directory is two levels up.
\newcommand{\olpath}{../../}

% load all the Open Logic definitions. This will also load the
% local definitions in open-logic-sample-config.sty
\input{\olpath/sty/open-logic.sty}

% we want all the problems deferred to the end
\input{\olpath/sty/open-logic-defer.sty}

% let's set the whole thing in Palatino, with Helvetica for
% sans-serif, and spread the lines a bit to make the text more
% readable

\usepackage{mathpazo}
\usepackage[scaled=0.95]{helvet}
\linespread{1.05}

\begin{document}

% First we make a titlepage

\begin{titlingpage}
\begin{raggedleft}
\fontsize{52pt}{2em}\selectfont\bfseries\sffamily
Sample\\[.5ex] 
Logic\\[.5ex] 
Text
\vskip 4ex
\normalfont\Huge\textbf{\href{http://openlogicproject.org/}{Open Logic Project}}

\end{raggedleft}

\vfill

% oluselicense generates a license mark that a) licenses the result
% under a CC-BY licence and b) acknowledges the original source (the
% OLP).  Acknowledgment of the source is a requirement under the
% conditions of the CC-BY license used by the OLP, but you are not
% required to license the product itself under CC-BY.

\oluselicense
% Title of this version of the OLT with link to source
{\href{https://github.com/OpenLogicProject/OpenLogic/tree/master/courses/sample}{\textit{Sample Logic Text}}}
% Author of this version
{\href{http://openlogicproject.org/}{OLP}}
\end{titlingpage}

\frontmatter
\pagestyle{ruled}

\tableofcontents*

\mainmatter

% olimport includes an entire part

\olimport*[sets-functions-relations]{sets-functions-relations}

% you can also import individual chapters

\part{First-order Logic}

\olimport*[first-order-logic/syntax-and-semantics]{syntax-and-semantics}

\olimport*[first-order-logic/models-theories]{models-theories}

\olimport*[first-order-logic/sequent-calculus]{sequent-calculus}

\olimport*[first-order-logic/completeness]{completeness}

% This allows you to leave out individual chapters, e.g.:

% \olimport*[first-order-logic/beyond]{beyond}

% But you can even import individual sections, in case you want to
% rearrange them or add text

\chapter{Beyond First-order Logic}

\olimport*[first-order-logic/beyond]{introduction}

\olimport*[first-order-logic/beyond]{many-sorted-logic}

\olimport*[first-order-logic/beyond]{second-order-logic}

\olimport*[first-order-logic/beyond]{higher-order-logic}

\olimport*[first-order-logic/beyond]{intuitionistic-logic}

\olimport*[first-order-logic/beyond]{modal-logics}

\olimport*[first-order-logic/beyond]{other-logics}

% OLEndChapterHook and OLEndPartHook should come at the end of each
% chapter/part.

\OLEndChapterHook

\OLEndPartHook

\olimport*[computability]{computability}

\olimport*[turing-machines]{turing-machines}

\olimport*[incompleteness]{incompleteness}

\stopproblems

% now typeset all the problems as an appendix. If you want problems at
% the end of each chapter, delete this part and put
% \problemsperchapter in the preamble

\backmatter

\chapter{Problems}

\printproblems

\end{document}

