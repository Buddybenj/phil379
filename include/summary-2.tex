A \emph{relation}~$R$ on a set~$X$ is a way of relating elements
of~$X$. We write $Rxy$ if the relation \emph{holds} between $x$
and~$y$. Formally, we can consider $R$ as the sets of pairs
$\tuple{x,y} \in X^2$ such that~$Rxy$. Being less than, greater than,
equal to, evenly dividing, being the same length as, a subset of, and the
same size as are all important examples of relations (on
sets of numbers, strings, or of sets). \emph{Graphs} are a general way
of visually representing relations. But a graph can also be seen as a
binary relation (the \emph{edge} relation) together with the
underlying set of \emph{vertices}.

Some relations share certain features which makes them especially
interesting or useful. A relation~$R$ is \emph{reflexive} if
everything is $R$-related to itself; \emph{symmetric}, if with $Rxy$
also $Ryx$ holds for any $x$ and $y$; and \emph{transitive} if $Rxy$
and $Ryz$ guarantees~$Rxz$. Relations that have all three of these
properties are \emph{equivalence relations}. A relation is
\emph{anti-symmetric} if $Rxy$ and $Ryx$
guarantees~$x=y$. \emph{Partial orders} are those relations that are
reflexive, anti-symmetric, and transitive. A \emph{linear order} is
any partial order which satisfies that for any $x$ and~$y$, either
$Rxy$ or $Ryx$. (Generally, a relation with this property is
\emph{connected}).

Since relations are sets (of pairs), they can be operated on as sets
(e.g., we can form the union and intersection of relations). We can
also chain them together (\emph{relative product}~$R \mid S$). If we
form the relative product of~$R$ with itself arbitrarily many times we
get the \emph{transitive closure}~$R^+$ of~$R$.
