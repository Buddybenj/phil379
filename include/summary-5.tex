A \emph{first-order language} consists of \emph{constant},
\emph{function}, and \emph{predicate} symbols. Function and constant
symbols take a specified number of arguments. In the \emph{language
  of arithmetic}, e.g., we have a single constant symbol~$\Obj 0$, one
1-place function symbol $\prime$, two 2-place function symbols~$+$ and
$\times$, and one 2-place predicate symbol~$<$. From \emph{variables}
and constant and function symbols we form the \emph{terms} of a
language. From the terms of a language together with its predicate
symbol, as well as the \emph{identity symbol}~$\eq$, we form the
  \emph{atomic formulas}. And in turn from them, using the logical
  connectives $\lnot$, $\lor$, $\land$, $\lif$, $\liff$ and the
  quantifiers $\lforall$ and $\lexists$ we form its formulas. Since
  we are careful to always include necessary parentheses in the
  process of forming terms and formulas, there is always exactly one
  way of reading a formula. This makes it possible to define things by
  induction on the structure of formulas.

Occurrences of variables in formulas are sometimes governed by a
corresponding quantifier: if a variable occurs in the \emph{scope} of
a quantifier it is considered \emph{bound}, otherwise
\emph{free}. These concepts all have inductive definitions, and we
also inductively define the operation of \emph{substitution} of a term
for a variable in a formula. Formulas without free variable
occurrences are called \emph{sentences}.

The \emph{semantics} for a first-order language is given by a
\emph{structure} for that language. It consists of a \emph{domain} and
elements of that domain are assigned to each constant symbol. Function
symbols are interpreted by functions and relation symbols by relation
on the domain. A function from the set of variables to the domain is
a \emph{variable assignment}. The relation of \emph{satisfaction}
relates structures, variable assignments and formulas;
$\Sat{M}[s]{!A}$ is defined by induction on the structure of~$!A$.
$\Sat{M}[s]{!A}$ only depends on the interpretation of the symbols
actually occurring in~$!A$, and in particular does not depend on $s$
if $!A$ contains no free variables. So if $!A$ is a sentence,
$\Sat{M}{!A}$ if $\Sat{M}[s]{!A}$ for any (or all)~$s$.

The satisfaction relation is the basis for all semantic notions. A
sentence is \emph{valid}, $\Sat{!A}$, if it is satisfied in every
structure. A sentence $!A$ is \emph{entailed} by set of
sentences~$\Gamma$, $\Gamma \Entails !A$, iff $\Sat{M}{!A}$ for all
$\Struct{M}$ which satisfy every sentence in~$\Gamma$. A set~$\Gamma$
is \emph{satisfiable} iff there is some structure that satisfies every
sentence in~$\Gamma$, otherwise unsatisfiable. These notions are
interrelated, e.g., $\Gamma \Entails !A$ iff $\Gamma \cup \{\lnot
!A\}$ is unsatisfiable.
